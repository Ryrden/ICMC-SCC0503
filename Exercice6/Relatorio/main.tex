\documentclass[a4paper, 12pt]{article}

\usepackage[utf8]{inputenc}
\usepackage{amsmath}
\usepackage{indentfirst}
\usepackage{graphicx}
\usepackage{multicol,lipsum}
\usepackage{hyperref}
\usepackage{minted}
\usepackage{cancel}

\begin{document}
%\maketitle

\begin{titlepage}
	\begin{center}
	
	%\begin{figure}[!ht]
	%\centering
	%\includegraphics[width=2cm]{c:/ufba.jpg}
	%\end{figure}

		\Huge{Instituto de Ciências Matemáticas e de Computação}\\
		\large{Departamento de Ciências de Computação}\\ 
		\large{SCC0503 - Algoritmos e Estruturas de Dados II}\\ 
		\vspace{15pt}
        \vspace{95pt}
        \textbf{\LARGE{Relatório Exercício 06}}\\
		%\title{{\large{Título}}}
		\vspace{3,5cm}
	\end{center}
	
	\begin{flushleft}
		\begin{tabbing}
			Alunos: Ryan Souza Sá Teles, Silmar Pereira da Silva Junior \\
            NUSP's: 12822062, 12623950.
			Professor: Leonardo Tórtoro Pereira\\
			%Professor co-orientador: \\
	\end{tabbing}
 \end{flushleft}
	\vspace{1cm}
	
	\begin{center}
		\vspace{\fill}
			 Julho\\
		 2022
			\end{center}
\end{titlepage}
%%%%%%%%%%%%%%%%%%%%%%%%%%%%%%%%%%%%%%%%%%%%%%%%%%%%%%%%%%%

\newpage
% % % % % % % % % % % % % % % % % % % % % % % % % %
\newpage
\tableofcontents
\thispagestyle{empty}

\newpage
\pagenumbering{arabic}
% % % % % % % % % % % % % % % % % % % % % % % % % % %
\section{Introdução}
A empresa BugSoft esta contratando quem for capaz de desenvolver um algoritmo que gere níveis no 
novo jogo deles, Credo dos Frita Sinos, munido de código de qualidade e sem nenhum bug foi 
desenvolvido um algoritimo para criação de dungeons usando Triangulação de Delaunay.

\newpage
\section{Desenvolvimento}

O desenvolvimento do projeto fez uso das abstrações já adotadas em sala de aula e o modelo de dígrafo selecionado para trabalhar na construção das dungeons foi o de "matriz de adjâcencias".\\

Tal decisão foi tomada pela comparação entre a complexidade das operações em comparação com a alternativa (lista de adjacências) e a frequência com que essas operações são feitas.\\
A principail operação realizada durante o algoritimo é a adição de vertices e arestas, em que a lista é O(1) em ambos, já a matriz é $O(|V|^2)$ e O(1) respectivamente. Com relação as demais ambas as abordagens tem complexidades semelhantes.\\
Outro fator que corroborou para a escolha foi a complexidade de espaço entre as opções, com a lista sendo $O(|V|+|E|)$ enquanto a matrix de adjacências sendo $O(|V|^2)$.\\

O algoritmo de Prim já implementado em sala de aula foi utilizado para "limpar" as arestas das dungeons recem criadas, de modo a garantir somente uma aresta entre uma sala e outra.

Para limpar as arestas, foi necessário utilizar o array de Predecessores gerado pelo algortimo de prim, com ele, foi implementado um metódo dentro de uma nova classe, GraphConverter, nela, foi gerado uma nova dungeon e alem disso, calculada a distancia entre cada sala por meio da distância euclidiana entre os pontos (vertíces).\\

O método para calculo de distância foi implementado na classe abstrata Abstract Graph, essa decisão foi feita devido a complexidade e quantidade de conflitos gerados ao abstrair uma classe chamada Dungeon, onde deveria conter tal método, em diversos pontos do código é chamado DigraphMatrix e GraphMatrix, gerando erros.\\


Com o objetivo de aplicar as boas práticas de código na atividade, uma nova package chamada dungeons foi criada para abrigar as classes referentes a criação de dungeon, deste modo, a organização do código se mantém.\\

Para a impressão da distância entre cada sala, foi utilizado o algoritmo Breadth First Search, o qual também foi implementado em sala de aula.

Abaixo, as implementações:\\

Implementação da classe GraphConverter (Classe qual possui o método que converte a lista de predecessores do algoritmo de prim para um grafo):


\begin{minted}[mathescape, linenos]{java}
public class GraphConverter {
    private GraphConverter() {}
    public static AbstractGraph predecessorListToGraph(AbstractGraph dungeon, int[] predecessor){
        AbstractGraph newDungeon = new DigraphMatrix();

        Room newRoom;
        for (int i = 0; i < dungeon.getNumberOfVertices(); i++) {
            newRoom = (Room)dungeon.getVertices().get(i);
            newDungeon.addVertex(newRoom);
        }

        float distance;
        Room source;
        Room destination;
        for (int i = 1; i < predecessor.length; i++) {
            source = (Room)dungeon.getVertices().get(predecessor[i]);
            destination = (Room)dungeon.getVertices().get(i);
            distance = dungeon.calcEuclidianDistance(source, destination);
            newDungeon.addEdge(source, destination, distance);
            newDungeon.addEdge(destination, source, distance);
        }
        return newDungeon;
    }
}
\end{minted}


\begin{minted}[mathescape, linenos]{java}
  
\end{minted}

Implementação do método calcEuclidianDistance:

\begin{minted}[mathescape, linenos]{java}
public float calcEuclidianDistance(Room source, Room destination) {
    int x1 = (int) source.getRoom().getX();
    int y1 = (int) source.getRoom().getY();
    int x2 = (int) destination.getRoom().getX();
    int y2 = (int) destination.getRoom().getY();
    return (float) Math.sqrt(Math.pow(x1 - x2, 2) + Math.pow(y1 - y2, 2));
}
\end{minted}

Implementação da Main (lógica do programa)

\begin{minted}[mathescape, linenos]{java}
public static void main(String[] args) {
    Scanner scanner = new Scanner(System.in);
    int randomSeed = scanner.nextInt();
    scanner.nextLine();
    int numberRooms = scanner.nextInt();

    AbstractGraph newDungeon = generateDungeonLevel(randomSeed, numberRooms);

    printDungeonWithBFS(newDungeon);
    SwingUtilities.invokeLater(() -> new DungeonGraphic(newDungeon).setVisible(true));
}
\end{minted}

Implementação dos submétodos da Main:

\begin{minted}[mathescape, linenos]{java}
private static void printDungeonWithBFS(AbstractGraph newDungeon) {
    var breadthFirstTraversal = new BreadthFirstTraversal(newDungeon);
    breadthFirstTraversal.traverseGraph(newDungeon.getVertices().get(0));
}

private static AbstractGraph generateDungeonLevel(int randomSeed, int numberRooms) {
    RandomSingleton.getInstance(randomSeed); //Inicializar gerador de números aleatórios
    var randomDungeonGenerator = new RandomDungeonGenerator(numberRooms); //Gerar 20 salas
    var dungeon = randomDungeonGenerator.getDungeon(); //Pegar o grafo com salas
    DelaunayTriangulation.triangulateGraphVertices(dungeon); //Criar arestas

    int[] predecessorIndexList = cleanEdgesAndGetPredecessorList(dungeon);

    return GraphConverter.predecessorListToGraph(dungeon, predecessorIndexList);
}

private static int[] cleanEdgesAndGetPredecessorList(AbstractGraph dungeon) {
    var primTraversal = new PrimMSTTraversal(dungeon);
    primTraversal.traverseGraph(dungeon.getVertices().get(0));
    return primTraversal.getPredecessorArray();
}
\end{minted}

\newpage
\section{Resultados}
\graphicspath{ {./Results/} }
Como resultados obtemos:

\begin{minted}{xml}
Room{(X, Y)= (48.0,84.0)} Distance: 0.0 
Room{(X, Y)= (217.0,127.0)} Distance: 174.38463 
Room{(X, Y)= (407.0,135.0)} Distance: 364.55298 
Room{(X, Y)= (352.0,201.0)} Distance: 328.33594 
Room{(X, Y)= (705.0,118.0)} Distance: 663.0375 
Room{(X, Y)= (743.0,209.0)} Distance: 761.6529 
Room{(X, Y)= (798.0,119.0)} Distance: 756.04285 
Room{(X, Y)= (790.0,301.0)} Distance: 864.96313 
Room{(X, Y)= (547.0,96.0)} Distance: 1008.0944 
Room{(X, Y)= (793.0,440.0)} Distance: 1003.9955 
Room{(X, Y)= (359.0,23.0)} Distance: 1209.7699 
Room{(X, Y)= (730.0,356.0)} Distance: 1108.9955 
Room{(X, Y)= (676.0,528.0)} Distance: 1150.3956 
Room{(X, Y)= (603.0,342.0)} Distance: 1236.7648 
Room{(X, Y)= (476.0,432.0)} Distance: 1372.2424 
Room{(X, Y)= (623.0,737.0)} Distance: 1366.011 
Room{(X, Y)= (382.0,302.0)} Distance: 1532.6669 
Room{(X, Y)= (360.0,410.0)} Distance: 1490.3102 
Room{(X, Y)= (509.0,744.0)} Distance: 1480.2257 
Room{(X, Y)= (313.0,364.0)} Distance: 1556.075 
\end{minted}

\end{document}