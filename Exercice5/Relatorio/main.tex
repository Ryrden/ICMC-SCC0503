\documentclass[a4paper, 12pt]{article}

\usepackage[utf8]{inputenc}
\usepackage{amsmath}
\usepackage{indentfirst}
\usepackage{graphicx}
\usepackage{multicol,lipsum}
\usepackage{hyperref}
\usepackage{minted}

\begin{document}
%\maketitle

\begin{titlepage}
	\begin{center}
	
	%\begin{figure}[!ht]
	%\centering
	%\includegraphics[width=2cm]{c:/ufba.jpg}
	%\end{figure}

		\Huge{Instituto de Ciências Matemáticas e de Computação}\\
		\large{Departamento de Ciências de Computação}\\ 
		\large{SCC0503 - Algoritmos e Estruturas de Dados II}\\ 
		\vspace{15pt}
        \vspace{95pt}
        \textbf{\LARGE{Relatório Exercício 04}}\\
		%\title{{\large{Título}}}
		\vspace{3,5cm}
	\end{center}
	
	\begin{flushleft}
		\begin{tabbing}
			Alunos: Ryan Souza Sá Teles, Silmar Pereira da Silva Junior \\
            NUSP's: 12822062, 12623950.
			Professor: Leonardo Tórtoro Pereira\\
			%Professor co-orientador: \\
	\end{tabbing}
 \end{flushleft}
	\vspace{1cm}
	
	\begin{center}
		\vspace{\fill}
			 Junho\\
		 2022
			\end{center}
\end{titlepage}
%%%%%%%%%%%%%%%%%%%%%%%%%%%%%%%%%%%%%%%%%%%%%%%%%%%%%%%%%%%

\newpage
% % % % % % % % % % % % % % % % % % % % % % % % % %
\newpage
\tableofcontents
\thispagestyle{empty}

\newpage
\pagenumbering{arabic}
% % % % % % % % % % % % % % % % % % % % % % % % % % %
\section{Introdução}
Após o recebimento do pedido, junto aos dados da BrabaLog foi desenvolvido um codigo com o intuito de representar os dados recebidos da malha na forma de um grafo, e ultilizar de tal estrutura para encontrar o melhor ponto para ser construido um centro de distribuição, e a cidade mais periférica que é atendida pela empresa. \\
O critério para encontrar o CCD será o conceito de vertice mais central, que é o vertice de menor distância ao seu vertice mais distânte.\\
Já, a cidade mais periferica será o oposto do critério para o CCD.\\

\newpage
\section{Desenvolvimento}

\newpage
\section{Resultados}
\graphicspath{ {./Results/} }


\end{document}



